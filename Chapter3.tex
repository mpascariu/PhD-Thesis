\documentclass[Thesis]{subfiles}
% --------------------------------------------

\begin{document}

\newpage
\pagenumbering{gobble}

\chapter{The Linear Link: Deriving Age-Specific Death Rates from Life Expectancy}
\pagecolor{pagecolor}\afterpage{\nopagecolor}
{
\vspace{2cm}
\large
Marius Dan Pascariu\\
Ugofilippo Basellini\\
Jose Manuel Aburto\\
Vladimir Canudas-Romo
}
\clearpage
\pagecolor{pagecolor}\afterpage{\nopagecolor}
\section*{}
\clearpage
% % --------------------------------------------


\begin{titlepage}
  \parbox{150mm}{
  \footnotesize
  Corresponding Author: Marius D. Pascariu\\[0mm]
  Institute of Public Health\\[0mm]
  University of Southern Denmark\\[0mm]
  J.B. Winslows Vej 9B, 5000 Odense, Denmark\\[0mm]
  E-mail: mpascariu@health.sdu.dk}
  \centering
	{\quad\par\vspace{2cm}}
	{\huge\bfseries The Linear Link: Deriving Age-Specific Death Rates 
	from Life Expectancy\par}
	{\large Short title: The Linear-Link Model\par}
	\vspace{2cm}
	{\Large Marius D. Pascariu\par}
	\emph{Institute of Public Health, University of Southern Denmark, Odense, Denmark}\\[2mm]
	\vspace{0.5cm}
	{\Large Ugofilippo Basellini\par}
	\emph{Institut national d'etudes d\'{e}mographiques (INED), Paris, France\\
	Institute of Public Health, University of Southern Denmark, Odense, Denmark\\[2mm]}
	{\Large Jos\'{e} Manuel Aburto\par}
	\emph{Institute of Public Health, University of Southern Denmark, Odense, Denmark}\\[2mm]
	{\Large Vladimir Canudas-Romo\par}
	\emph{School of Demography, The Australian National University, Canberra, Australia}
	\vfill

% Bottom of the page
	{\large 15 December 2017\par}
\end{titlepage} 
\newpage


\quad\vspace{4cm}
\pagenumbering{arabic}\setcounter{page}{50}
\section*{Abstract}
The prediction of human longevity levels in the future by direct forecasting of life expectancy offers numerous advantages, compared to methods based on extrapolation of age-specific death rates. However, the reconstruction of accurate life tables starting from a given level of life expectancy at birth, or any other age, is not straightforward. Model life tables have been extensively used for estimating age patterns of mortality in poor-data countries. We propose a new model inspired by indirect estimation techniques applied in demography, which can be used to estimate full life tables at any point in time, based on a given value of life expectancy at birth. The methods presented in this paper are implemented in a publicly available \texttt{R} package.
%(JEL: C20, C130, C53, C800, C87)


\subsection*{Keywords:}
\emph{Indirect estimation; Life expectancy; Forecasting; Death rates; Age-patterns of mortality}

\newpage
\section{Introduction}\label{sec:Intro}

Understanding human mortality dynamics is of utmost importance in the context of rapid ageing process together with the increase in length of life experienced by most populations nowadays. The link between the pension systems sustainability and changes in life expectancy is more apparent than ever in light of the recent reforms that are taking place in Europe. In countries like Germany and Finland the level of retirement benefits are linked to life expectancy, in other countries like the U.K. and France the retirement age is set to increase from the current levels and implicitly the contribution period for pensions to be extended as people live longer \citep{stoeldraijer2013}.

In predicting demographic processes, such as human mortality, methods involving extrapolation of mortality rates or probabilities are the most common approaches. Stochastic models, such as those proposed by \cite{lee1992} or \cite{cairns2006} have gained significant popularity and have been extensively used in the last two decades. Ideas that focus only on life expectancy have given rise to a new approach. The models introduced by \cite{torri2012}, \cite{raftery2014} and \cite{pascariu2017} are partially inspired by the linear time trends observed in life expectancy at birth in many developed countries, particularly in the second half of the twentieth century \citep{oeppen2002, white2002}. These life expectancy models are very appealing because they offer the same, or higher, level of forecast accuracy in terms of life expectancy but with the advantage of being parsimonious, focusing on one variable rather than several. They rely on a measure that incorporates all the factors that influence longevity (lifestyle, access to healthcare, diet, economical status, etc.), namely life expectancy \citep{christensen1996}. Furthermore, highly aggregated data by age provide valuable information that can be used to tackle the issue of mortality forecasting from a clearer perspective. The U.S. Census Bureau predicts the future mortality levels up to year 2100 based on projections of life expectancy at birth by sex and race, modelling an exponential decline of the gap to the observed upper asymptote of life expectancy. The period age-specific death rates are estimated in a subsequent step using these projections \citep{uscensus2014}.

Transformation of life expectancy into mortality rates at every age can be accomplished by exploiting the regularities of age patterns of mortality. In actuarial science, the use of life tables and other models reflecting life contingencies is motivated by the need to determine insurance and pension risks, net premiums, and benefits. Basically, actuarial methods combine the life table with functions related to an assumed rate of interest \citep{moller2007, dickson2013}. Based on the relevance of having a set of age-specific death rates, we propose a method to create such an array of values from one available life expectancy. 

Our method extends the work initiated by the different systems of model life tables  (\citealp{gabriel1958, UN1955}, \citeyear{UN1967}; \citealp{coale1966}, \citeyear{coale1983};  \citealp{ledermann1969, sullivan1972}); Brass' relational model (\citeyear{brass1968, brass1971}) and the recent extensions of techniques for estimating age patterns of mortality by \cite{murray2003} and \cite{wilmoth2012}. Our model is also related to the work of \cite{mayhew2013} that uses the trends in life expectancy to establish a robust statistical relation between changes in life expectancy and survivorship. A further, similar approach to the one developed here is that of \cite{sevcikova2016} which incorporates a method based on the Lee--Carter model for converting projected life expectancies at birth to age-specific death rates in the UN's 2014 probabilistic population projections.

Relational models were developed for estimation purposes in poor-data contexts. These models rely on parameters that depict the relationships between various measures of age-specific and overall mortality. The parameters in a relational model are estimated from an initial analysis of historical mortality data and become fixed thereafter. Once those values have been estimated, the model simplifies to a few initial inputs like: reported child survival, records of population growth, responses to questions about fertility and mortality and in our case life expectancy at birth or at any other age. Furthermore, in recent years the accessibility of historical mortality data, such as the Human Mortality Database (HMD), means that the necessary information to estimate the parameters of relational models is readily available. As shown here, an algorithm that derives a life table based only on life expectancy at birth can also be widely used in forecasting practice. 

The remainder of the article is organized as follows. First, in Section \ref{sec:DataMethods} a new model to derive age-specific death rates is introduced and a description of the data used in testing is provided. Section \ref{sec:Results} shows computed results and illustrations of life expectancy decomposition into death rates in several populations. The discussion and conclusion are presented in Section \ref{sec:Discussion}.

\section{Data and Methods}\label{sec:DataMethods}

\subsection{Data}\label{sec:Data}
The data source used in this article is the Human Mortality Database (\citeyear{hmd2017b}), which contains historical mortality data for homogeneous populations in 43 different countries and territories. The HMD constitutes a reliable data source because it includes high quality data that were subject to a uniform set of procedures, thus maintaining the cross-national comparability of the information.

In order to test and illustrate the performance of the method, we fit the model using the death rates computed using death counts and population exposed to the risk of death in the calendar year for the female populations of the England \& Wales, France, Sweden and USA available in the HMD. The reason for using these death rates is that old age mortality in the HMD is often subject to diverse correction procedures and modelling depending on the country \citep{wilmoth2007}. 

The reconstructive power of the method for a point forecast of life expectancy is demonstrated using the 1980--2014 mortality data between age 0 and 100. Data at higher ages might be unreliable or too sparse for different populations, which would make it difficult to differentiate between data related problems or modelling issues. To compute the accuracy measures and the estimation errors, the 1965--90 data is applied to the same age range.


\begin{figure}[!t]
{\centering\includegraphics[width=1\linewidth]{./figures/Chapter3/Figure1-1}} 
\caption{\textit{Linear relation between life expectancy at birth and death-rates on a log-log scale, by age. The axis are labelled in normal scale for better interpretability. Based on HMD mortality data starting from 1980 for 43 countries and territories.}}\label{fig:Figure1}
\end{figure}


\subsection{The Model}\label{sec:Methods}

Given a predicted level of life expectancy the age pattern of mortality can be derived using a linear relation. The logarithm age-specific death rate at time $t$, denoted $m_{x,t}$, can be expressed as a linear function of the logarithm of life expectancy at a given age $\theta$, denoted $e_{\theta,t}$. Formally:

\begin{equation}\label{eq:L1}
\log{m_{x,t}} = \beta_x\log{e_{\theta,t}} + \varepsilon_{x,t} \quad \textrm{for} \quad x \geq \theta,
\end{equation} 
where $x$ can take values between 0 and  $\omega$, the highest attainable age, and $\beta_{x}$ can be regarded as an age-specific parameter. Parameter $\varepsilon_{x,t}$ denotes a set of normally distributed errors with mean zero and variance $\sigma^2$. For example, when $\theta$ equals zero, we estimate an entire mortality curve based on life expectancy at birth using this equation; when $\theta > 0$, we estimate the mortality curve starting from age $\theta$.

The method presented here combines the strong linear relations found when comparing life expectancies and age-specific death rates on a log-log scale. Figures \ref{fig:Figure1} and \ref{fig:Figure2} show those relations, although the slopes and intercepts vary between ages, in all cases there is a strong linear concordance between the level of overall mortality, as depicted by life expectancy, and the individual age-specific death rates. These relations have been key in much of the work on model life tables (\citealp{gabriel1958, UN1955}, \citeyear{UN1967}; \citealp{coale1966}, \citeyear{coale1983};  \citealp{ledermann1969}). Inspired by the two-dimensional system (age and time) of the Log-quadratic model \cite{wilmoth2012} and the strong linear trends in Figures \ref{fig:Figure1} and \ref{fig:Figure2}, we derive the age pattern of mortality based on a given value of life expectancy, e.g. forecasted life expectancy value, and a matrix of age-specific death rates from the past.

This model can be seen as a method that links the life expectancy at age $\theta$ at any point in time to a mortality curve estimated from the death rates $m_x$'s that return a life expectancy level of $e_\theta$. Therefore we will refer to it as the linear-link (LL) model. To gain precision in the fitting of the death rates the LL model can be extended by including additional parameters:

\begin{equation}
 \begin{split}
	\log{m_{x,t}} & = \beta_x\log{e_{\theta,t}} + \nu_xk + \varepsilon_{x,t} 
	\quad \textrm{for} \quad  x \geq \theta, \\
	& \sum_{x = \theta}^{\omega} \nu_{x} = 1, 
	\quad
	\textrm{and} \quad \nu_{x} \geq 0,
 \end{split}
\end{equation}

where $\nu_{x}$ is the speed of mortality improvement over time at age $x$, $k$ is an estimated correction factor independent of time and $\varepsilon_{x,t}$ are independent and identically distributed random variables normally distributed with mean zero and variance $\sigma^2$. Different than the Log-quadratic model that has a fixed set of parameters for any input value, here the parameters $\beta_x$, $\nu_x$ and $k$ can be calculated for each set of age-specific death rates and future life expectancy. Thus, it can be seen as an extension of the log-quadratic model for countries that have good quality data, where an entire life table is completed from one target value of life expectancy.


\begin{figure}[!t]
{\centering \includegraphics[width=1\linewidth]{./figures/Chapter3/Figure2-1}} 
\caption{\textit{Linear relation between life expectancy at age 65 and death-rates on a log-log scale, by age. The axis are labelled in normal scale for better interpretability. Based on HMD mortality data starting from 1980 for 43 countries and territories.}}\label{fig:Figure2}
\end{figure}

In addition, the LL model is closely related to the LC model. Indeed, if one sets the parameters  $\beta_x log{e_{\theta,t}} = \alpha_x$, $\nu_x = \beta_x$, and $k = k_t$, we obtain the LC model. Despite their similarities, there are two important differences between the two models. First, while the shape of the mortality pattern $\alpha_x$ is constant in the LC model, the first term of the LL changes with the level of life expectancy considered; as such, there exists a range of different baseline mortality curves of the LL model depending on the particular level of $e_\theta$. Second, the $k$ parameter is not modelled as a function of time, instead the parameter is used as an optimization variable affecting the shape of the age pattern of mortality to achieve the desired target life expectancy $e_\theta$. Thus, the $k$ parameter enhances the flexibility of the method and the accuracy of the results.

\subsection{Algorithm}
Let $t$ be an observed unit of time in the interval $\lbrace 1,...,T \rbrace$ and $\tau$ be a an unobserved point in time $T+n$ e.g. a date in the future. The objective is to convert a value of life expectancy, $e_{\theta,\tau}^{*}$, into a schedule of age-specific death rates $m_{x,\tau}^{*}$. The level of life expectancy can be a predicted value given by certain extrapolation method or the target values resulted following a subjective judgement. Input data will be a collection of observed death rates $m_{x,t}$ and a level of life expectancy $e^{*}_{\theta,\tau}$. The steps involved in the algorithm to obtain the desired death rates are the following:

\begin{enumerate}
\item Using the Kannisto mortality model (see Appendix) extend $m_{x,t}$ to higher age groups up to age $\omega$ for all times $t$. The highest attainable age, $\omega$, can be set for example to 120.

\item Estimate the slope of the linear relation between life expectancy and the death-rates, $\beta_x$, over the observation time $t$. This is done by using the method of the least squares approach, by minimizing the sum of squared residuals:

\begin{equation}
\sum_{x} { \left[ \log{m_{x,t}} - \beta_x \log{e_{\theta,t}} \right]}^{2} = \sum_{x}{\left[\varepsilon_{x,t} \right]^2}.
\end{equation}

Alternatively, the parameters of the model can be estimated by assuming that deaths follow a Poisson distribution \citep{brouhns2002}, $D_x \backsim Poisson(E_x \cdot m_{x,t})$, with $m_{x,t} = \exp(\beta_x\log{e_\theta} + \nu_xk)$. In order to use this approach death counts ($D_{x,t}$) and exposure data ($E_{x,t}$) are needed. Sensitivity analysis shows that the difference between the two fitting procedure return minor discrepancies (see section \ref{sec:PoissonMLE} in the Appendix for more details).

\item Estimate the parameter $\nu_{x}$ by computing the singular value decomposition (SVD) of the matrix of regression residuals, $\textbf{R}$, obtained in the previous step,

\begin{equation}
SVD\left[ \textbf{R} \right] ={\textbf{DPQ}}^{T}={d}_{1}{p}_{1}{q}_{1}^{T}+... ,
\end{equation}

where 
\begin{equation*}
\textbf{R} = \begin{bmatrix}
\varepsilon_{0,1}& \varepsilon_{0,2}& \cdots & \varepsilon_{0,T} \\
\varepsilon_{1,1}& \varepsilon_{1,2}& \cdots & \varepsilon_{1,T} \\
\vdots & \vdots & \ddots & \vdots\\
\varepsilon_{\omega,1}& \varepsilon_{\omega,2} & \cdots & \varepsilon_{\omega,T}\\
\end{bmatrix},
\end{equation*}


\begin{figure}[!t]
{\centering \includegraphics[width=1\linewidth]{./figures/Chapter3/Figure3-1}}
\caption{\textit{Estimated parameters of the Linear--link model, using HMD data from 1980 to 2014 and life expectancy at birth $(\theta = 0)$.}}\label{fig:Figure3}
\end{figure}

$\textbf{P}=[p_{1},p_{2}, ...] $ and $\textbf{Q}=[q_{1},q_{2}, ...]$ are matrices of left and right singular vectors, and $\textbf{D}$ is a diagonal matrix with singular values along the diagonal. The fist term of the $SVD$, $d_{1}p_{1}q^{T}_{1}$, is used for obtaining the estimates of $\nu_x$. Parameter $\nu_x$ can be interpreted as the rate of mortality improvement over age.  

\item Smooth the $\beta_{x}$ and $\nu_{x}$ parameters using splines. This step is important to obtain graduated mortality curves and avoid projecting age-specific noise in the jump-off life table. However, if the graduation is not of interest or if the input data-set is large enough, this step can be skipped.

\item Compute the initial mortality rates\footnotemark \text{ by} $m_{x,\tau}^{*} = \exp \{ \beta_x\log{e^{*}_{\theta,\tau}}+ \nu_{x}k \}$, where $k=0$. 

\footnotetext{The change in age-specific death rates can be assumed to be constant over time, in which case the fitted $\nu_x$ is used in computing $m_x$. Or, a shift in the speed of improvement can be imposed by "rotating" the $\nu_x$ coefficients. For more details see Section \ref{sec:rotateVx} in the Appendix.}

\item Optimize the mortality curve given in the previous step by finding the value of $k$ where the difference between target life expectancy $e_{\theta,\tau}^{*}$ and an estimated life expectancy $e_{\theta,\tau}$ is below a tolerance level, for example 0.001. Where $e_{\theta,\tau}$ represents the level of life expectancy at birth computed based on the mortality rates obtained in step (5). Usually $k$ will be in the range of $\left( -150, +150 \right)$ depending on the length of the forecast window.
\end{enumerate}


\begin{figure}[!b]
{\centering \includegraphics[width=1\linewidth]{./figures/Chapter3/Figure4-1}}
\caption{\textit{Observed and estimated death rates for female populations in 2014. Computed based on mortality data in the period 1965--90.}}
\label{fig:Figure4}
\end{figure}

The estimated $\beta$ parameters for the female populations in England \& Wales, France, Sweden and USA, exhibit minor differences between the countries, and capture well the important stages of human mortality: the decreasing infant mortality, the accidental hump, the adult mortality characterized by an exponential increase with age and, finally, a mortality plateau above the age of 100 years. As shown in Figure \ref{fig:Figure3}, the $\nu_x$ pattern differs from population to population. In the case of Sweden, a larger variance is observed over ages due to a smaller population size and more significant changes at younger ages in the period analysed. 

\begin{figure}[!t]
{\centering \includegraphics[width=1\linewidth]{./figures/Chapter3/Figure5-1}}
\caption{\textit{Mean absolute errors (\%) of the estimated log-death rates against the actual log-death rates between 1991 and 2014. Computed based on female mortality data in the period 1965--90.}}
\label{fig:Figure5}
\end{figure}


\section{Results and Illustration}\label{sec:Results}

We perform back-testing against the observed mortality for the female populations living in England \& Wales, France, Sweden and USA. We take the period of 1965--90 as reference and use the death-rates and life expectancies at birth in this time interval to fit our model. Based on single values of life expectancy at birth observed in the subsequent years we derive complete mortality curves. For example, the estimation of the age-specific death rates in 2014 is demonstrated in Figure \ref{fig:Figure4}. The reconstructed mortality curves are in general smoother than the observed data; this is more evident in the case of Sweden, where the population is smaller compared with the other three countries.

\begin{figure}[!b]
{\centering \includegraphics[width=1\linewidth]{./figures/Chapter3/Figure6-1}}
\caption{\textit{Relative errors (\%) of the estimated log-death rates against the actual log-death rates between 1991 and 2014. Computed based on female mortality data in the period 1965--90.}}
\label{fig:Figure6}
\end{figure}

Figure \ref{fig:Figure5} shows that the average relative error of the estimated log-death rates, compared to the actual rates between 1991 and 2014, is between 0.9\% and 3.8\%. It can be also noted that the longer the prediction interval, the larger the errors. In the case of female populations living in England \& Wales, France, Sweden and USA, the largest errors occurred in 2014; nonetheless these are smaller than 3.8\% of the actual log-death rate. This value is an average over the entire comparable age range (0--100). The largest impact on the overall accuracy occurs at advanced ages, where the level of uncertainty is higher. Figure \ref{fig:Figure6} offers a view of the error distribution by age and time. However, the life expectancy at birth computed based on the estimated death rates matches exactly the actual life expectancy in the respective year and country. 


\begin{figure}[!t]
{\centering \includegraphics[width=1\linewidth]{./figures/Chapter3/Figure7-1}}
\caption{\textit{Comparison of the mortality curves predicted by Lee--Carter and Linear-Link models in 2040 from female populations. The models are fitted on the 1980--2014 historical period.}}
\label{fig:Figure7}
\end{figure}

In order to test the conversion reliability of a forecast value of life expectancy, we compare the results generated by the LL model against the predicted mortality from the Lee--Carter model (\citeyear{lee1992}). 

The LC model is fitted over the 0--95 age-range using the historical data from 1980 to 2014, and used to forecast death rates 26 years in the future, until 2040. The estimated matrix of predicted death rates between age 0 and age 95 is extended up to age 120 using the Kannisto model (see equation (\ref{eq:kannisto1}) in the Appendix). If multiple projections are simulated for the same forecast point, the LC would produce a range of outcomes that can be translated into life expectancies using standard life table calculations. Any predicted life expectancy given by LC is used as an input value in the LL model to derive the mortality curve, thus obtaining two comparable curves. For every simulated trajectory, the LL method can produce a mortality curve, generating the uncertainty around the median prediction. Figure \ref{fig:Figure7} shows that the reconstruction method employed by the LL model gives an almost coincident mortality curve when compared with the LC curve for female populations in 2040. The 99\% prediction intervals are computed based on ten thousand Monte-Carlo simulations. Besides the LL model showing a more smooth age-pattern when compared with the LC results, it can perfectly estimate the predetermined life expectancy.  

\section{Discussion}\label{sec:Discussion}

We have introduced a simple method, the Linear-link model, to derive the entire schedule of age-specific death rates, based on a single value of life expectancy and prior knowledge of human mortality patterns. The model is based on the observed linearity between age-specific death rates, $m_x$, and life expectancy at a certain age, $e_\theta$. The model can be regarded as a decomposition approach of the human mortality curve between the general age pattern, $\beta_x$, and an age-specific speed of improvement, $\nu_x$. The method is inspired by: (1) the Log-quadratic model \citep{wilmoth2012} in the sense of using a leading indicator in determining the age pattern of mortality; (2) the model introduced by \cite{sevcikova2016} by adopting an inverse approach to death rates estimation starting from life expectancy; (3) the Lee--Carter model (\citeyear{lee1992}) using the same interpretation of mortality improvement over time and age; and finally (4) the \cite{li2013} method to model the rotation of age patterns of mortality decline for long-term projections.

The method can be useful in three different situations: future target life expectancy, life tables for countries with deficient data and historical life table construction. The former is the one explored in the present manuscript, while the latter two are only briefly discussed since their development goes beyond the scope of the paper of presenting the LL model. 

First the model can be used in forecasting practice when the level of life expectancy is forecast first. We showed that this model can accurately reconstruct a Lee--Carter forecast starting from a single value of life expectancy at birth. This is important, because the Linear-link model offers the possibility of taking advantage of the more regular pattern of the life expectancy evolution. It is much easier and parsimonious, from a technical perspective to forecast one time series of expectation of life than to extrapolate 100 or 110 series of death probabilities corresponding to each age group. In the same manner adult mortality can be estimated based on a value of life expectancy at an advanced age, say age 65. In Figure \ref{fig:Figure2}, we showed that the linearity between death rates at advance ages and life expectancy at age 65 on a log scale is maintained. A greater variation is observed only at advanced ages, above 100, where data is sparse in general. 

Second, the method can be used to build model life tables and estimate the current age patterns of mortality in poor-data countries or regions, like Sub-Saharan Africa. In this case the parameters of the model are estimated based on a collection of historical life tables from several regions or populations. Once the parameters have been estimated, and implicitly the model life table, they remain fixed. The relevant mortality curve is simply calibrated in accordance with a single value of life expectancy at birth or any other age instead of child mortality like in the case of \cite{wilmoth2012}. In our analysis we show examples using high quality data from developed countries in order to demonstrate the efficiency of the model, and to be able to asses the accuracy of the mortality curve reconstruction. However, the estimation procedure and the steps of the algorithm are the same for this case too.

Third, the LL model can be a useful tool in a variety of research contexts
of historical demography like backward projections and estimation of mortality levels in historical populations. Due to the existence of scarce non-standardized population data in the past and population census only for the more recent times, the very possibility of projecting mortality backward is of theoretical interest \citep{ediev2011}.

According to our analysis, the optimal number of years to be used in the fitting of the model is between 30 and 35 years. If a longer time interval was used, the parameter estimates would lose their relevance. For example, the present rate of improvement in the death rates is different from that experienced 50 years ago, because of fundamental changes in society and scientific advances during this period \citep{bengtsson2006,rau2008}. In the same manner over a longer period of time the linearity between life expectancy and death rates might be challenged, however this should be investigated from case to case.

The speed of improvement in age-specific death rates over ages changes over time. For example, in recent decades, a faster pace of improvement was observed at ages 65 and above \citep{vaupel1997, shkolnikov2011}. We address the possibility of experiencing accelerating or decelerating speeds of mortality improvements over different age ranges by assigning different weights to the estimated $\nu_x$ curve when the life expectancy at birth continues to advance over age 75. The effect of this method can be best observed in Figure \ref{fig:Figure7} in the case of France. Under the implicit assumption of constant mortality improvements, the LC forecast generates a second mortality hump around age 50. The estimated mortality curve given by the LL model has a less pronounced effect due to the rotated $\nu_x$ parameter. See a detailed description of the method in Appendix \ref{sec:rotateVx}.  

The evolution of human mortality is a complex process that is driven by a large number of factors and can not be explained by a single statistical model. The Linear-link method offers an alternative approach to deriving the unknown levels of mortality in the future. In contrast with methods like the Lee--Carter model that extrapolate age-specific rates or probabilities directly the method presented here recognizes life expectancy as the main driver of mortality at any given age and employs an indirect estimation algorithm. These methods can complement each other and help us understand better the future longevity experienced by populations.  

 
\newpage

\section{Appendix}\label{Appendix}
\subsection{The Kannisto Model}\label{sec:Kannisto}

Normally, mortality data is available in tables that contain detailed information up to age 85, 100 or 110, with last age group being open. In order to extend the mortality rates up to age 120, the Kannisto method \citep{thatcher1998} for old-age mortality with an asymptote equal to one can be employed: 

\begin{equation}
\label{eq:kannisto1}
  m_x = \dfrac{\alpha e^{\beta \chi}}{1+ \alpha e^{\beta \chi}}, 
\end{equation}

which can also be written as a linear function of age

\begin{figure}[!b]
{\centering \includegraphics[width=0.9\linewidth]{./figures/Chapter3/Figure8-1}}
\caption{\textit{Extension of female mortality rates using the Kannisto model in 2014.}}
\label{fig:Figure8}
\end{figure}


\begin{equation}
\label{eq:kannisto2}
  logit(m_x) = ln(\alpha) + \beta \chi + \epsilon_{\chi},
\end{equation}
where $\epsilon$ is a normally distributed variable with mean zero, $\chi = x - 80$, and parameters $\alpha$ and $\beta$ are positive real numbers. The model is usually fitted between age 80 and 95.

Assuming that $D_x \backsim Poisson(E_x \cdot m_x (\alpha , \beta))$ the parameters $\alpha$ and $\beta$ can be derived by maximizing the log-likelihood function:

\begin{equation}
\label{eq:kannistoL}
  log L(\alpha , \beta) = \sum_{x=80}^{95} \Big\{
  D_x log \big[m_x(\alpha , \beta) \big] - 
  E_x m_x(\alpha , \beta) \Big\} + constant,
\end{equation}

where $D_x$ denotes the number of deaths that occurred at age $x$, $E_x$ represents the population exposed to risk at the same age, and $m_x$ is the age-specific death rate.

The Kannisto model is not only useful to obtain values for oldest-old mortality but also to smooth the rates computed on smaller sample sizes. The case of Sweden presented in Figure \ref{fig:Figure8} can be relevant here, where in 2014 the number of females aged 100 and above was less than 1,700. A small sample size  can create difficulties in obtaining reliable mortality estimates based only on empirical observations. Outliers are expected to show up  from year to year. 

%\newpage
\subsection{Maximum likelihood estimation}
\label{sec:PoissonMLE}

Assuming that deaths are Poisson distributed, the LL model can be fitted by maximising the log-likelihood given by 

\begin{equation}
\label{eq:MLE}
  log L(a ,\nu,k) = \sum_{x,t} \Big\{
  D_{x,t}(a_x + \nu_xk) - 
  E_{x,t}^{c} \exp(a_x + \nu_xk) \Big\} + constant,
\end{equation}
where $a_x = \beta_x\log{e_{\theta,t}}$.
The parameters are estimated following an updating scheme proposed by \citet{brouhns2002} based on the Newton-Raphson algorithm. The updating procedure, with initial values $\hat{a}_x(0)=0,\hat{\nu}_x(0)=1$, and $\hat{k}(0)=0$, is as follows:

\begin{align*}
  \hat{a}_x(w+1) & = \hat{a}_x(w) - \frac{\sum_t(D_{x,t}-\hat{D}_{x,t}(w))}{-\sum_t\hat{D}_{x,t}(w)}, \\
  \quad \hat{\nu}_x(w+1)& =\hat{\nu}_x(w), \\
  \quad \hat{k}(w+1) & = \hat{k}(w), \\
  \hat{k}(w+2) & = \hat{k}(w+1) - \frac{\sum_x(D_{x,t}-\hat{D}_{x,t}(w+1))\hat{\nu}_x(w+1)}{-\sum_x\hat{D}_{x,t}(w)(\hat{\nu}_x(w+1))^2}, \\
  \quad \hat{a}_x(w+2) & =\hat{a}_x(w+1), \\
  \quad\hat{\nu}_x(w+2) & = \hat{\nu}_x(w+1), \\
  \hat{\nu}_x(w+3) & = \hat{\nu}_x(w+2) - \frac{\sum_t(D_{x,t}-\hat{D}_{x,t}(w+2))\hat{k}(w+2)}{-\sum_t\hat{D}_{x,t}(w+2)(\hat{k}(w+2))^2}, \\
  \quad \hat{a}_x(w+3) & =\hat{a}_x(w+2), \\
  \quad\hat{k}(w+3) & =\hat{k}(w+2),
\end{align*}

where $\hat{D}_{x,t}(w) = E_{x,t}^{c} \exp(\hat{a}_x(w) + \hat{\nu}_x(w)\hat{k}(w))$, is the estimated number of deaths after iteration $w$. After the parameters are estimated, the parameter $a_x$ is transformed using the LL model, $a_x = \beta_x\log{e_{\theta,t}}$.

The maximum likelihood estimation (MLE) has several advantages over least squares (OLS) and SVD methods or even weighted least squares (WLS) used in \cite{wilmoth2007}. Several reasons have been given in the literature \citep{brouhns2002, alho2000}. One example would be the increasing confidence intervals by age. This is because in the OLS estimation via SVD the errors are assumed to be homoskedastic and normally distributed, which is quite a heavy assumption. The logarithm of the observed force of mortality is much more variable at older ages than at younger ages because of the much smaller absolute number of deaths at older ages. Therefore, since the number of deaths is a counting variable, the Poisson assumption seems more reasonable \citep{brillinger1986}.
 
However, in order to use this approach we need death counts $D_{x,t}$ and exposures $E_{x,t}$, which are not always available. This being the reason why the model described by equation (\ref{eq:L1}) is chosen in the article. Our methodology is targeting populations with deficient data as well as populations whose estimates of mortality rates are provided without disaggregation by deaths and exposures. Although conceptually a Poisson setting would be better, the SVD approach is a pragmatic decision for practical reasons. As shown in figure \ref{fig:Figure9} the difference between the two estimation methods for the case of England \& Wales females is very small but the data requirements is higher in the case of MLE.
 
\begin{figure}[!ht]
{\centering \includegraphics[width=0.9\linewidth]{./figures/Chapter3/Figure9-1}}
\caption{\textit{Comparison of the fitted mortality curves and parameter estimates of the Linear-Link model using the OLS+SVD and MLE fitting procedures. England \& Wales female data for 1980--2014 period is used.}}
\label{fig:Figure9}
\end{figure}


%\newpage
\subsection{Rotation of mortality improvements} 
\label{sec:rotateVx}

One of the main limitations of the LC model (\citeyear{lee1992}) is the central assumption of constant rates of mortality declines at different ages, resulting from the time-invariant $b_x$ coefficient of age-specific mortality improvements \citep{bongaarts2005}. The assumption has been violated in several low-mortality countries in recent decades, because  rates of mortality improvements have tended to decline over time at younger ages, and they have risen at older ages \citep{kannisto1994,vaupel1998,wilmoth1999}. 

It is important to take into consideration the changing age pattern of mortality improvements to produce more accurate mortality forecasts, and projection methodologies that ignore such rotation will lead to errors, particularly in the projected age patterns of future death rates \citep{li2013}. \citeauthor{li2013} proposed an extension of the LC method to incorporate the rotation of the age patterns of mortality decline for long-term projections. 

\begin{figure}[!b]
{\centering \includegraphics[width=1\linewidth]{./figures/Chapter3/Figure10-1}}
\caption{\textit{Assumption of the change in $\nu_x$ pattern following the increase in life expectancy at birth from 75 to 102 years.}}
\label{fig:Figure10}
\end{figure}

Here, we propose a modification of the original \cite{li2013} methodology that ensures the rotation of the rate of mortality improvement over age in the LL model, $\nu_x$. The methodology is composed of two different steps.

First, we derive an ultimate schedule of mortality improvements, $\nu^u_x$,  from the estimated coefficient $\nu_x$. In particular, the ultimate rates of improvement between ages 0 and 65 are set equal to the average improvement at adolescent and adult ages (15--65); from age 65 onwards, improvements decrease following a logistic shape, and they converge to zero at age 130. 

Second, we smooth the transition from $\nu_x$ to $\nu^u_x$ using the weight function proposed by \cite{li2013}. The transition, and therefore the degree of rotation of $\nu_x$, is dependent on $e^*_0(\tau)$, the predicted value of life expectancy at birth (an input in our LL model). Formally, the weight function $w_s$ can be expressed as:
\begin{equation}
\label{Eq:ws}
  w_s(\tau)= \left \{ \frac{1}{2} \left [ 1 + \mathrm{sin}  \left [ \frac{\pi}{2} \left ( 2w(\tau) - 1 \right ) \right ]  \right ] \right \}^p \quad \textrm{with} \quad w(t) = \frac{e^*_0(\tau)-80}{e^u_0-80} \, .
\end{equation}

Because $\nu_x$ parameter is scaled in order to take values between 0 and 1 the ultimate pattern of mortality improvement by age will be the same for all countries. If the scaling process is ignored, the same pattern is obtained in all cases with a different level of $\nu_x$ between age 0 and 65. The estimated death rates would be the same in both case because of the adjustment provided by $k$ parameter.  

The power of the smooth-weight function, $p$, regulates the speed of the rotation. It varies between 0 and 1, and lower values correspond to faster rotations for levels of $e^*_0(\tau)$ closer to 80. The level of life expectancy at which the rotation finishes, $e^u_0$, is also arbitrary; here, we follow the recommendations of \cite{li2013} and set the intermediate value of 0.5 for $p$ and the age 102 for $e^u_0$.

The rotated coefficient of mortality improvement over age, denoted $N^r_x(\tau)$, can thus be written as: 
\begin{equation}
\label{Eq:Nrx}
  N^r_x(\tau) = \left\{ 
  \begin{array}{lll}
  \nu_x \, , \quad &  e^*_0(\tau) < 80 \, , \\
  \left [ 1- w_s(\tau) \right ]\nu_x + w_s(\tau) \nu^u_x \, , \quad & 80 \leq e^*_0(\tau) < e^u_0 \, , \\
  \nu^u_x \, , \quad & e^*_0(\tau) \geq e^u_0 \, . \\
  \end{array}
  \right.
\end{equation}


\end{document}

