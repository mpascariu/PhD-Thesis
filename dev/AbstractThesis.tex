\documentclass[12pt]{article}\usepackage[]{graphicx}\usepackage[]{color}

%% maxwidth is the original width if it is less than linewidth
%% otherwise use linewidth (to make sure the graphics do not exceed the margin)
\makeatletter
\def\maxwidth{ %
  \ifdim\Gin@nat@width>\linewidth
    \linewidth
  \else
    \Gin@nat@width
  \fi
}
\makeatother

\definecolor{fgcolor}{rgb}{0.345, 0.345, 0.345}
\newcommand{\hlnum}[1]{\textcolor[rgb]{0.686,0.059,0.569}{#1}}%
\newcommand{\hlstr}[1]{\textcolor[rgb]{0.192,0.494,0.8}{#1}}%
\newcommand{\hlcom}[1]{\textcolor[rgb]{0.678,0.584,0.686}{\textit{#1}}}%
\newcommand{\hlopt}[1]{\textcolor[rgb]{0,0,0}{#1}}%
\newcommand{\hlstd}[1]{\textcolor[rgb]{0.345,0.345,0.345}{#1}}%
\newcommand{\hlkwa}[1]{\textcolor[rgb]{0.161,0.373,0.58}{\textbf{#1}}}%
\newcommand{\hlkwb}[1]{\textcolor[rgb]{0.69,0.353,0.396}{#1}}%
\newcommand{\hlkwc}[1]{\textcolor[rgb]{0.333,0.667,0.333}{#1}}%
\newcommand{\hlkwd}[1]{\textcolor[rgb]{0.737,0.353,0.396}{\textbf{#1}}}%
\let\hlipl\hlkwb

\usepackage{framed}
\makeatletter
\newenvironment{kframe}{%
 \def\at@end@of@kframe{}%
 \ifinner\ifhmode%
  \def\at@end@of@kframe{\end{minipage}}%
  \begin{minipage}{\columnwidth}%
 \fi\fi%
 \def\FrameCommand##1{\hskip\@totalleftmargin \hskip-\fboxsep
 \colorbox{shadecolor}{##1}\hskip-\fboxsep
     % There is no \\@totalrightmargin, so:
     \hskip-\linewidth \hskip-\@totalleftmargin \hskip\columnwidth}%
 \MakeFramed {\advance\hsize-\width
   \@totalleftmargin\z@ \linewidth\hsize
   \@setminipage}}%
 {\par\unskip\endMakeFramed%
 \at@end@of@kframe}
\makeatother

\definecolor{shadecolor}{rgb}{.97, .97, .97}
\definecolor{messagecolor}{rgb}{0, 0, 0}
\definecolor{warningcolor}{rgb}{1, 0, 1}
\definecolor{errorcolor}{rgb}{1, 0, 0}
\newenvironment{knitrout}{}{} % an empty environment to be redefined in TeX

\usepackage{alltt}
\usepackage[a4paper, margin=2.6cm]{geometry}
\usepackage{hyperref}
\definecolor{mycol}{RGB}{8, 29, 88}
\hypersetup{
    colorlinks = true,          % false: boxed links; true: colored links
    urlcolor = mycol          % color of external links
}

\renewcommand{\baselinestretch}{1.3}
\pagenumbering{gobble}
\date{}

\begin{document}
	
{\centering\large PhD Thesis Summary\par}
\vspace{1cm}
{\centering\Large\bfseries Modelling and forecasting mortality\par}
\vspace{1cm}
{\centering\large Marius D. Pascariu\par}
\vspace{1cm}

For the world as a whole, life expectancy has more than doubled over the past two centuries. This transformation of the duration of life has greatly enhanced the quantity and quality of people’s lives. It has fuelled enormous increases in economic output and in population size, including an upsurge in the number of elderly. Understanding human mortality dynamics is of utmost importance in the context of rapid ageing and increasing length of life experienced by most populations nowadays. The present thesis highlights new and innovative methods for estimating and projecting future mortality levels among humans.

Three studies have been devised, which develop and analyse relevant statistical models for addressing uncertainty in future mortality. The studies are in the form of research manuscripts that are/will be published in scientific journals together with software packages ensuring the reproducibility of the results. In the first study, a method for forecasting life expectancy for females and males is developed. To forecast female life expectancy, the method is based on the analysis of the gap in life expectancy between females in a given country and females in record-holding countries. To forecast male life expectancy, the gap between male life expectancy and female life expectancy in a given country is analysed. In the second study, we explore a new approach inspired by indirect estimation techniques applied in demography, which can be used to estimate full life tables at any point in time, based on a given value of life expectancy at birth or at any other age. The third study makes use of the statistical properties of a probability density function in order to estimate the distribution of deaths of a population in the future. We employ time series methods for forecasting a limited number of central statistical moments and then reconstruct the future distribution of deaths using the predicted moments. The estimation of the density function is done using the maximum entropy approach.  

The results show that mortality modelling can be tackled from different perspectives and higher accuracy of the future trajectories can be obtained when compared with the more traditional extrapolative methods based of age specific death rates or probabilities.

\end{document}
